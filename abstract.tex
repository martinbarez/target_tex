% +--------------------------------------------------------------------+
% | Copyright Page
% +--------------------------------------------------------------------+

\newpage

\thispagestyle{empty}

\begin{center}

{\bf \Huge Abstract}

  \end{center}
\vspace{1cm}
In recent years there has been a resurgence in the space race, motivated especially by commercial companies. Their aircrafts are equipped with a multitude of sensors, one of them being hyperspectral cameras. This type of camera takes images in hundreds of different spectral bands, with the aim of providing information of the ground.
\\
Because of the large size of these images, they are sent to ground stations for processing, with the consequent cost of transmission and storage. Preferably these images should be processed or compressed on site and only a fraction of the data obtained should be sent. Given the spatial environment and the characteristics of these types of algorithms, FPGAs or ASICs are postulated as an optimal system for their implementation.
\\
This work presents an FPGA implementation of the Reed-Xiaoli algorithm of anomaly detection for hyperspectral images. For its implementation, an analysis of the operations of the algorithm has been made, centered in a floating point version and another one in integer arithmetic, and of the repercussions that certain decisions have with the precision that is reached.
Thus, demonstrating how complex algorithms with floating point operations can be executed in FPGAs by transforming them to use integer arithmetic.

\vspace{1cm}

\begin{center}

{\bf \Large Keywords}

   \end{center}

   \vspace{0.5cm}
   
   keywords
   


