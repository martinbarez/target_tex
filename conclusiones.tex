\cleardoublepage
\chapter{Conclusion}
%\label{ch:chapter1}
\label{makereference}
El procesado de imágenes hyperespectrales es una herramienta muy potente que facilita desde operaciones de minería, seguimiento de objetivos o análisis de contaminación. El avance tecnológico de las cámaras acentúa la necesidad de realizar este tipo de análisis a bordo y con ello el uso de plataformas especificas como las FPGAs.
\\
\\
El análisis completo de este tipo de imágenes no resulta siempre viable y algoritmos de detección de objetivos como el aquí expuesto permiten no solo una notable reducción en los requisitos de ancho de banda si no que también permiten el uso de este tipo de sensores en aplicaciones a tiempo real.
\\
\\
Los resultados obtenidos son positivos ya que muestran una reducción en el uso de recursos frente a otras implementaciones anteriores gracias a su aproximación a través del uso de lógica en punto fijo. El uso de una plataforma reconfigurable permite además ajustar la precisión de este sistema incluso después de la puesta en marcha de este.