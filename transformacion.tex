\cleardoublepage
\chapter{Transofrmación de logica}
%\label{ch:chapter1}
\label{makereference}
\section{kdfkd}
creo que debería tener un apartado explicando las diferencias entre arimeticas
Como está codificada cada una, ejemplos de uso de logica, y ya dentro un subapartado con la estrategia de transformacion

\section{estrategia de transformación de aritmética entera}
Una vez hecha la implementación en software se cambiará el tipo de datos de punto flotante a entero con 64 bits. Con este cambio se podrán observar en que pasos del algoritmo se pierde precisión por llegar a los limites representables con enteros con esa resolución -tanto por ser números cercanos al cero como al desbordar por llegar a los valores máximos y mínimos- y se intentará mejorar la precisión desplazando, es decir multiplicando y dividiendo por potencias de dos. Así, si un valor se acerca a 0 será multiplicado para mantener más precisión y si se encuentra cerca del desbordamiento, será dividido para evitarlo. Al ser los resultados de cada pixel relativos al resto, el orden de anomalidad no sé ve afectado siempre que estás operaciones se apliquen a todo el conjunto de datos de manera simultanea. Además, estas operaciones resultan triviales en un afpga.
Conforme se vaya mejorando la precisión, también se irá limitando la cantidad de bits usada con el obejtivo de usar menos bits en la fpga y ahorrar logica. Esto es de especial importancia en los dsp, donde al ser bloques definidos en fabricacion, tienen operandos de tamaños fijos. Aquí se muestra una tabla con el número de DSP que necesita una multiplicación según la precisión de sus operandos.
Con los datos obtenidos de precisión y de uso de tejido, se elegirá una precisión para cada operación manteniendo estos dos valores en equilibrio.
Aquí toca mencionar, que la operación con diferencia más sensible a la perdida de precisión e introducción de errores es la inversa, específicamente la división. Aquí se ha tenido especial cuidado tanto que se realizan diferentes desplazamientos según el paso en el que se realiza esta división.

\section{validación y precisión}
Cuando pueda pongo un poco lo que me has dicho