% +--------------------------------------------------------------------+
% | Copyright Page
% +--------------------------------------------------------------------+

\newpage

\thispagestyle{empty}

\begin{center}

{\bf \Huge Resumen en castellano}

  \end{center}
\vspace{1cm}

En los últimos años ha ocurrido un resurgimiento de la carrera espacial motivado especialmente por empresas comerciales. Sus aeronaves son equipadas con una multitud de sensores, siendo uno de ellos las cámaras hiperespectrales. Este tipo de cámaras toma imágenes en cientos de bandas espectrales diferentes, con el objetivo de proporcionar información sobre el terreno.
\\
A causa del gran tamaño de las imágenes hiperespectrales, estas son enviadas a la Tierra para su procesado, con el consecuente coste de transmisión y almacenamiento. Preferentemente estas imágenes deberían procesarse o comprimirse in situ para enviar solo una fracción de los datos obtenidos. Dados el entorno espacial y las características este tipo de algoritmos, las FPGAs o ASICs se postulan como un sistema óptimo para su implementación.
\\
Este trabajo presenta una implementación sobre FPGAs del algoritmo Reed-Xiaoli de detección de anomalías para imágenes hiperespectrales. Para su implementación se ha realizado un análisis de las operaciones del algoritmo, centrada en una versión en punto flotante y otra en aritmética de enteros, y de las repercusiones que tienen ciertas decisiones con la precisión que se alcanza.
Demostrando de esta manera cómo algoritmos complejos con operaciones en punto flotante pueden ser ejecutados en FPGAs al transformarlos para utilizar aritmética de enteros.



\vspace{1cm}


\begin{center}

{\bf \Large Palabras clave}

   \end{center}

   \vspace{0.5cm}
   
Imágenes hiperespectrales, Algoritmo RX, Aritmética de punto flotante, Aritmética de enteros, Hardware reconfigurable, VHDL.   


